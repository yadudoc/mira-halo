\documentclass[conference,10pt]{IEEEtran}
\usepackage{fancyhdr}
\usepackage{amssymb}
\usepackage{amsmath}
\usepackage{amsfonts}
\usepackage[T1]{fontenc} % get tt fonts to work right
\usepackage{graphicx}
\usepackage{multirow}
\usepackage{color}
\usepackage{caption}
\DeclareCaptionType{copyrightbox} % workaround for bug in caption
\usepackage{subcaption}
\usepackage{xspace}
\usepackage{url}

\begin{document}

\special{papersize=8.5in,11in}
\setlength{\pdfpageheight}{\paperheight}
\setlength{\pdfpagewidth}{\paperwidth}


\title{Mapping HALO Exchange onto Toruses and Stuff}

\author{\IEEEauthorblockN{
Yadu Nand\IEEEauthorrefmark{1}\IEEEauthorrefmark{2}\IEEEauthorrefmark{3},
Timothy G. Armstrong\IEEEauthorrefmark{1}}
  \IEEEauthorblockA{
  \IEEEauthorrefmark{1}Dept. of Computer Science,
    University of Chicago,
    Chicago, IL, USA}
  \IEEEauthorblockA{\IEEEauthorrefmark{2}Mathematics and Computer Science Division,
    Argonne National Laboratory,
    Argonne, IL, USA}
  \IEEEauthorblockA{\IEEEauthorrefmark{3}Computation Institute,
    University of Chicago and Argonne National Laboratory,
    Chicago, IL, USA}
}

\maketitle

 
\begin{abstract}
Abstract goes here
\end{abstract}

\section{Introduction}

Halo exchange is a common communication pattern in parallel codes, where
each process is assigned an application subdomain and must periodically
communicate with other processors that have neighboring subdomains to
update information about the state of the boundary between subdomains.
A common special case is when a multi-dimensional cartesian space is
decomposed into subdomains of equal size.  For example, in the three-dimensional
case, a 8x8x8 cube might be decomposed into 256 2x1x1 cubes for execution on 256 processors.

This paper explores the problem of mapping such multi-dimensional cartesian
halo exchange communications onto parallel computers with hypercube or
torus networks.

\section{Example Section}

We're going to cite Swift/T~\cite{SwiftT_2013} and include an illustration
(see Figure~\ref{fig:task-data}).

\label{sect:ddt-model}
\begin{figure}
  \center
  \includegraphics[width=0.325\textwidth]{fig/task-data}
  \caption{This is a figure.
    \label{fig:task-data}}
\end{figure}


\label{sect:Analytic-model}
\begin{figure}
  \center
  \includegraphics[width=0.325\textwidth]{fig/Analytic_regular_no_congestion_170_304_304.png}
  \caption{This is a figure.
    \label{fig:Analytic_regular_no_congestion_170_304_304.png}}
\end{figure}

\section{High-Performance Computer Networks}

\subsection{Blue Gene/Q 5D torus}
RedBook~\cite{BGQ_RedBook_2013}

\subsection{Cray Gemini 3D torus}

\section{Models for Network Communication}

\section{Experimental Design}

\section{Results}

\section{Conclusion}


\bibliographystyle{abbrv}
\bibliography{halo}

\end{document}
